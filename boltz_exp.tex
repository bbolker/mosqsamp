\documentclass{article}
\title{explaining the Maxwell-Boltzman expression}
\usepackage{amsmath}

\newcommand{\multinom}{{\cal M}}

\begin{document}

In the main document I say

\begin{quote}
Then the likelihood of observing an occupancy spectrum $S$ is
\begin{equation}
P(S|B,M) =  \frac{1}{B^M} \multinom(S) \multinom(V)
\end{equation}
where $\multinom(V)$ can also be written as
$M!/\prod_i (i!)^{s_i}$.
\end{quote}

Where does this come from?

\begin{itemize}
\item The total number of ways in which $M$ (\emph{distinct})
mosquitos can bite $B$ birds is $B^M$. Each bird could be bitten
by any combination of mosquito $a$, mosquito $b$, mosquito $c$ 
\ldots e.g. for $B=3$, $M=2$, we have 
$\{\{ab,0,0\}, \{0,ab,0\}, \{0,0,ab\}, \{a,0,b\},\{a,b,0\},
\{0,a,b\},\{0,b,a\},\{b,0,a\},\{b,a,0\}$ (9 combinations).
(But easier to see this by assigning birds to mosquitos. Each
mosquito must bite exactly one of the $B$ birds, and birds \emph{can}
be bitten more than once: for this we have $\{A,A\},\{A,B\},\{A,C\},
\{B,A\}, \ldots$.)
\item the total number of configurations of number of times each
bird is bitten is $\multinom(V)$, because the total number of permutations
of $M$ objects ($\sum v_i$) is $M!$, and the number of ways of
reordering those within the particular partition $V$ is
$\prod v_i!$
\item since this represents one particular ordering (e.g. in 
$V=\{3,1,1,0\}$ the bird bitten 3 times could be any one of the
4 birds, as could the bird bitten 0 times), we also have to allow
for reorderings of birds. This is $multinom(S)$, because we have
$B=\sum s_i$ objects, the total number of permutations is $B!$,
and the number of ways of reordering them within the partition $S$
is $\prod s_i!$.
\end{itemize}

Let's take the example from the document. $B=4$, $M=5$, 
$V=\{3,1,1,0\}$, $S=\{1,2,0,1\}$.

\begin{equation*}
\begin{split}
P(S|B,M) & =  \frac{1}{B^M} \multinom(S) \multinom(V) \\
         & =  \frac{1}{4^5} \multinom(\{3,1,1,0\}) \multinom(\{1,2,0,1\}) \\
         & =  \frac{1}{1024} \cdot 20 \cdot 12 \\
         & =  \frac{15}{64} = 0.234375.
\end{split}
\end{equation*}
I don't really want to write down the 240/1024 configurations!
\end{document}

